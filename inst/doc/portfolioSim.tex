\documentclass{article}

\pagestyle{headings}

\usepackage{Sweave}

\DefineVerbatimEnvironment{Sinput}{Verbatim}{fontsize=\small,fontshape=sl}
\DefineVerbatimEnvironment{Soutput}{Verbatim}{fontsize=\small}
\DefineVerbatimEnvironment{Scode}{Verbatim}{fontsize=\small,fontshape=sl}

\begin{document}

\title{Performing equity investment simulations with the \texttt{portfolioSim} package}
\author{Kyle Campbell, Jeff Enos, and David Kane}

%%\VignetteIndexEntry{Performing equity investment simulations with the portfolioSim package}
%%\VignetteDepends{portfolio}

\maketitle


\setcounter{secnumdepth}{3}


\begin{abstract}
\label{abstract}

The \texttt{portfolioSim} package provides a flexible system for
back-testing the performance of investment strategies using historical
data.  The package is designed to take into account the limitations of
day-to-day trading, so that a simulation's performance will resemble
as closely as possible the performance that a given strategy would
actually have achieved over a given period of time.

\end{abstract}

\section{Introduction}

Note: this document is unfinished and is a work-in-progress.\\

Whatever investment strategy one uses to manage a portfolio, perhaps
the most basic and most important question one can ask of it is: ``How
well does my strategy work?''  Whenever one uses a systematic model to
make financial decisions, one would like to have some assurance that
the model will yield positive returns on an investment.
Unfortunately, due to the unpredictability of the market, no estimates
of a model's performance will ever be absolutely certain.  The only
sure indicator of an investment's value is the returns one can measure
after the fact.

Therefore, barring the ability to foresee the future course of the
market, the best way to guage a model's accuracy is to examine its
track record.  Using historical data, it is possible to calculate how
much money one would have made or lost had one invested according to a
given strategy over some past period of time.  This technique, known
as ``backtesting,'' is widely used by financial professionals to test
investment strategies.  By making the assumption that there is at
least some correlation between a model's past and future performance,
investors can use the results of a backtest to make an informed
decision on whether to use a strategy in their future investments.

The \texttt{portfolioSim} package is intended to give investors the
tools to make such a decision.  It does so by simulating a changing
portfolio of investments over some period in the past.  The simulator
makes use of the \texttt{portfolio} package\footnote{See Enos and
Kane, Analysing equity portfolios in R, for an introduction to the
\texttt{portfolio} package.} to manage the portfolio.  The simulator
takes historical market data supplied by the user and follows an
investment strategy, also provided by the user, to determine which
stocks to trade and when to trade them.  At the end of the simulation,
the user is provided with detailed information on the portfolio's
performance over that period.  From this information, the user can
determine whether or not the investment strategy was effective.

The \texttt{backtest} package\footnote{See Campbell, Enos, Gerlanc,
and Kane, Conducting Backtests in R, for more information on the
\texttt{backtest} package.} can be used to conduct one specific type
of simple backtest for a single type of portfolio, specifically, a
long-short portfolio formed from the highest and lowest ranked stocks
of an input signal.  The \texttt{portfolioSim} package, in contrast,
is much more flexible, allowing the user to specify virtually any
criteria for constructing and maintaining a portfolio.

In addition to being far more flexible in both the input it can
receive and the output it returns, the \texttt{portfolioSim} package
has several major advantages over the \texttt{backtest} package.  One
of the most important is its ability to step through time, period by
period, making investments based on many market variables as they
stood at some point in the past.  The \texttt{backtest} package
reduces the problem of forming a portfolio to a single ranking
variable, buying the highest ranked stocks and shorting the lowest.
But in reality, there are often many other factors that an investor
must take into account.  It may take several days to trade to a
desired portfolio, and by the time all the selected stocks are
actually acquired the prices may have changed and the stocks may no
longer be desirable.  The cost of making trades must be taken into
account as well, along with the turnover in the portfolio.

The \texttt{backtest} package ignores all of these complications, and
thus the results it reports, while useful for gauging the accuracy of
a stock ranking, do not accurately reflect the returns one could
expect to gain from actual investments.  The \texttt{portfolioSim}
package, on the other hand, is so named because it tries to simulate,
as closely as possible, the day-to-day trading process as it occurs in
reality.  It allows the user to take into account daily trading
volume, stock prices, trade-cost adjustment, portfolio equity, and
many other such considerations that the \texttt{backtest} package
ignores.


\section{Running a simulation}

At its core, the \texttt{portfolioSim} package is a very simple
machine for maintaining a representation of a changing portfolio over
some span of time.  At any moment, this portfolio consists of holdings such as:

\begin{Schunk}
\begin{Soutput}
  id shares
1  A     10
2  B     20
3  C    -10
\end{Soutput}
\end{Schunk}

Changes to a portfolio can be made in the form of ``trades''.

\begin{description}

\item{\bf{trade}}: A transaction which changes the current holdings of
the portfolio by means of buying, selling, shorting, or covering
stocks.

\end{description}

Consider the follow list of trades:

\begin{Schunk}
\begin{Soutput}
  id side shares
1  B    S     20
2  C    C      5
3  D    B     10
4  E    X     20
\end{Soutput}
\end{Schunk}

Applying these four trades to the holdings above would transform the
portfolio into a new set of holdings, shown below:

\begin{Schunk}
\begin{Soutput}
  id shares
1  A     10
2  C     -5
3  D     10
4  E    -20
\end{Soutput}
\end{Schunk}

Essentially, running a simulation consists of exposing a portfolio to
different sets of trades at different points in time and observing the
results.  The duration of time over which the simulation is conducted
is divided up by the user into discrete ``periods'', at each of which
a new set of trades is performed.  The manner in which trades are
selected for a given period is how we define an investment strategy.
This strategy is supplied by the user in the form of an interface
passed to the simulator.

Because there are no restrictions on how the user sets up an
investment strategy, it is possible that the trades interface will
return a set of trades that cannot be performed for one reason or
another.  For example, if a stock has an average daily trading volume
of one thousand shares and one of the trades returned by the trades
interface is to buy two thousand shares of that stock, the simulator
will have to know that it can only expect to fill part of this trade
in a single period.

In order to check for such limitations, the simulator needs to have
access to a fair amount of information on all the securities that
might be traded in each period.  Specifically, the simulator needs to
know the daily trading volume for each security, the price of each
security at the start and the close of the trading day, and the
returns from each security over the given period.  All of this data
must be supplied by the user, in the form of another interface passed
to the simulator.  This interface is queried at the start of every
period, and provides the simulator with all the data it will need
during that period.

These two interfaces, along with a list of periods, are the primary
forms of input the simulator requires from the user.  When a
simulation is run, the simulator goes period by period, processing the
following steps at each period:

\begin{enumerate}

\item{\bf{Query data interface}}: First, the simulator retrieves
all the data pertaining to the period.  This includes the price,
volume, and return information noted above, as well as any other
information that might be required by the interfaces.  For example,
the trades interface may require some additional information to help
it determine the set of trades to make.  Where this data is stored
depends entirely on the data interface.  It could be saved in the
interface itself, in a local database, or over a network.

\item{\bf{Clean up current holdings}}: Next, the simulator compares
the holdings currently in the portfolio to the new data returned by
the data interface.  If there are any holdings which are no longer in
the investable universe (for example, holdings in delisted
securities), then the simulator attempts to remove those holdings from
the portfolio.

\item{\bf{Query trades interface}}: Next, the trades interface is
used to obtain a list of desirable trades.  All the details of trade
selection that affect which trades receive priority, such as
trade-cost adjustment, must be handled within the trades interface.
The simulator will change the trades returned from the interface only
if the number of shares to be traded exceeds some set percentage of
the daily trading volume.  All other considerations must be dealt with
by the interface itself.  The \texttt{stiFromSignal} interface
included in the \texttt{portfolioSim} package is a highly flexible
interface that takes into account many of considerations one would use
to make real trades, including trade-cost adjustment, target equity,
and trading volume (see section 4 for more details).

\item{\bf{Save start data}}: For each period, the simulator saves
three types of results: data on the portfolio at the start of the
period, data on the trades made during the period, and data on the
portfolio at the end of the period.  The user has some flexibility
about which types of data should be saved, but the most basic type of
results for the starting portfolio consists of the long and short
equity and the size of the long and short sides of the portfolio.

\item{\bf{Save period data}}: After the results for the start
portfolio are saved, the simulator saves the results for the period
itself.  This consists of the turnover in the portfolio, the turnover
in the universe of investable stocks, and the performance of the
portfolio during this period.  The performance is calculated based on
the returns of the stocks in the portfolio at the \emph{start} of the
period.  In other words, the performance for the portfolio resulting
from trades in one period is calculated during the following period.
In addition, the user can select to save details on the performance of
individual stocks, and/or the list of trades to be performed during
this period.

\item{\bf{Perform trades}}: Once the start and period results have
been saved, the simulator exposes the portfolio to the final list of
trades for the period.  First, the trades are checked to makes sure
they do not exceed the fixed fill volume percentage; any trades that
do are adjusted to match the maximum allowed percentage of the daily
trading volume.  Then the holdings in the portfolio are transformed
according to the list of trades.

\item{\bf{Save end data}}: Finally, the simulator saves the results
for the portfolio at the end of the period.  Again, this consists
primarily of saving the size and equity of both sides of the
portfolio.

\end{enumerate}

These steps are repeated for each period in the simulation, with the
portfolio carrying over from one period to the next.  After all the
periods have been processed, the simulator is left with summary
information on each step of the portfolio's history, from which it is
easy to calculate the overall performance of the portfolio.


\subsection{A simple example}

Consider a basic simulation conducted over four periods in a market
with only four stocks: ``A'', ``B'', ``C'', and ``D''.  The prices of
these stocks remain constant over the course of the simulation, and
the daily trading volume is 100 shares for each stock.  For the
purposes of this example, we ignore the implementation of the data and
trades interfaces.  We begin the simulation with no holdings.

First, the simulator gets the data for the first period:

\begin{Schunk}
\begin{Soutput}
  period id ret start.price end.price volume universe
1      1  A   0          10        10    100     TRUE
2      1  B   0          10        10    100     TRUE
3      1  C   0          10        10    100     TRUE
4      1  D   0          10        10    100     TRUE
\end{Soutput}
\end{Schunk}

Since the current portfolio is empty, we have no holdings to compare
to the data.  Next, the simulator gets a list of trades from the
trades interface:

\begin{Schunk}
\begin{Soutput}
  id side shares
1  A    B     10
2  B    X     10
\end{Soutput}
\end{Schunk}

For the first period, we want to perform two very simple trades:
buying 10 shares of stock ``A'' and shorting 10 shares of stock ``B''.
Both of these amounts are lower than the maximum fill volume
percentage (15 percent), so the simulator can carry out the complete
list of trades.  Our new holdings after the first period are thus:

\begin{Schunk}
\begin{Soutput}
  id shares
1  A     10
2  B    -10
\end{Soutput}
\end{Schunk}

The simulator then proceeds to the next period after saving out the
summary data for the first period.  New data and a new set of trades
are retrieved from the interfaces.  We now have a slightly more
complex set of trades to be performed:

\begin{Schunk}
\begin{Soutput}
  id side shares
1  B    C      5
2  C    B      5
3  A    X     10
4  A    S     10
\end{Soutput}
\end{Schunk}

During this period, we want to cover some of the shares of stock ``B''
which we had shorted in the previous period; we want to buy into stock
``C'', and we want to switch our holdings of stock ``A'' from the long
side to the short side of the portfolio.  Again, all of these trades
are less than 15 shares, so the simulator can fill them all.  Our
holdings after the second period are:

\begin{Schunk}
\begin{Soutput}
  id shares
1  A    -10
2  B     -5
3  C      5
\end{Soutput}
\end{Schunk}

In the trades for the third period, however, not all of the trades can
be filled during a single period:

\begin{Schunk}
\begin{Soutput}
  id side shares
1  B    C      5
2  D    B     20
\end{Soutput}
\end{Schunk}

The trades interface has returned a list of trades that includes
buying 20 shares of stock ``D''.  However, the daily trading volume
for this stock is 100 shares, of which the simulator will fill a
maximum of 15\%.  Therefore, only 15 shares of stock ``D'' will be
bought during this period.

\begin{Schunk}
\begin{Soutput}
  id shares
1  A    -10
3  C      5
4  D     15
\end{Soutput}
\end{Schunk}

Note also that we have covered the rest of our shares in stock ``B'',
so it no longer appears in our holdings.

Finally, in the last period, the trades interface returns only one
trade: the remaining 5 shares we had intended to buy of stock ``D''.
Note that the simulator does not automatically process left over
trades during the following period, but this is something that a good
trades interface will take into account.

\begin{Schunk}
\begin{Soutput}
  id side shares
1  D    B      5
\end{Soutput}
\end{Schunk}

Looking at the data for period 4, we see that stock ``C'' is no longer
in the universe of investable stocks:

\begin{Schunk}
\begin{Soutput}
   period id ret start.price end.price volume universe
13      4  A   0          10        10    100     TRUE
14      4  B   0          10        10    100     TRUE
15      4  C   0          10        10    100    FALSE
16      4  D   0          10        10    100     TRUE
\end{Soutput}
\end{Schunk}

Because our current holdings include shares of stock ``C'', the
simulator will attempt to remove these shares from the portfolio
before processing the trades for this period.  Therefore, our updated
portfolio before we perform the final set of trades contains these
holdings:

\begin{Schunk}
\begin{Soutput}
  id shares
1  A    -10
4  D     15
\end{Soutput}
\end{Schunk}

Finally, we buy the remaining 5 shares of stock ``D''.  There are no
more periods, so the simulation is finished.  Our final holdings are:

\begin{Schunk}
\begin{Soutput}
  id shares
1  A    -10
2  D     20
\end{Soutput}
\end{Schunk}

This example covers only the most basic functionality of the
simulator.  There are many other features built into the
\texttt{portfolioSim} package, including the ability to calculate
exposures and contributions across different variables.  Most
importantly, the implementation of the different interfaces allows the
user include build many custom features into the simulator.


\section{Overview of the \texttt{portfolioSim} package}

Of the classes contained in the \texttt{portfolioSim} package, the
user interacts primarily with objects of the \texttt{portfolioSim}
class, used to conduct the simulation, and objects of the
\texttt{simResult} class, used to store and analyze the results of the
simulation.  In addition, there are three interface classes which the
user can choose to implement in order to customize the simulation to
meet specific needs.

\subsection{The \texttt{portfolioSim} class}

When beginning a new simulation, the first step is to construct an
object of class \texttt{portfolioSim} which will contain all the
information required by the simulator.  An instance of class
\texttt{portfolioSim} represents a unique simulation, which can then
be run at any time by calling the \texttt{runSim} method.  This allows
the user to make changes to the simulator after a run and, over
repeated simulations, to see how those changes affect the results.

A \texttt{portfolioSim} object contains the following slots:

\begin{itemize}

\item{\texttt{periods}}: A data frame listing the periods to be used
in the simulation.  Each period represents a single iteration of the
simulator, in which a new set of trades is calculated and carried out.
The periods data frame must have columns \texttt{period},
\texttt{start}, and \texttt{end}.  The \texttt{period} column contains
labels which are used throughout the simulator to represent each
period.  The \texttt{start} and \texttt{end} columns are used to
differentiate between saved data from before and after the trades are
performed in each period.  Generally, these columns should contain the
actual dates corresponding to each period.

\begin{Schunk}
\begin{Soutput}
  period               start                 end
1      1 2006-01-01 09:30:00 2006-03-31 16:00:00
2      2 2006-04-01 09:30:00 2006-06-30 16:00:00
3      3 2006-07-01 09:30:00 2006-09-30 16:00:00
4      4 2006-10-01 09:30:00 2006-12-31 16:00:00
\end{Soutput}
\end{Schunk}

\item{\texttt{freq}}: The annual frequency of the periods listed in
the periods slot.  For example, the frequency corresponding to the
periods data frame shown above is be 4.  When running a simulation
with monthly periods, the frequency should be 12.  With daily periods,
it should be 252, the total number of trading days in a year.

\item{\texttt{data.interface}}: A data interface object of some class
containing the virtual class \texttt{simDataInterface}.  The data
interface serves to transform the raw data used in the simulation into
an object of class \texttt{simData}, containing information on a
single period.

\item{\texttt{trades.interface}}: A trades interface object of some
class containing the virtual class \texttt{simTradesInterface}.  The
trades interface represents the implementation of the trading stategy
to be tested in the simulation.  Based on the current portfolio and
the data available for a given period, the trades interface contains
some mechanism for determining a set of trades to make.  These trades
are encapsulated in a \texttt{simTrades} object which the interface
returns to the simulator.  The default trades interface is an object
of class \texttt{stiFromSignal}, which uses some signal to rank stocks
and form a portfolio based on those rankings.

\item{\texttt{summary.interface}}: An optional summary interface
object of a class containing the virtual class
\texttt{simSummaryInterface}.  The summary interface allows the user
to specify information to be saved out during the simulation beyond
that supported by the result classes \texttt{instantData} and
\texttt{periodData}.

\item{\texttt{start.holdings}}: A portfolio object representing the
portfolio at the start of the simulation.  If this slot is not
specified, the simulator starts with an empty portfolio.  See the
documentation in the \texttt{portfolio} package for information on
constructing a portfolio.

\item{\texttt{fill.volume.pct}}: The maxiumum percentage of the daily
trading volume of a stock that the simulator is allowed to trade in
a single period.  The default is 15.

\item{\texttt{exp.var}}: A character vector of additional variables to
be used when analyzing the exposures for each period.  See section 7
for more information.

\item{\texttt{contrib.var}}: A character vector of additional
variables to be used when analyzing the contributions for each period.
See section 7 for more information.

\item{\texttt{out.loc}}: The location of a directory, relative to the
current working directory, to which the simulator will save out the
results.  The \texttt{runSim} method will also return the full
\texttt{simResult} object, but for large simulations it is more
efficient to let the simulator save out to a directory and then load
in the desired results after the simulation has finished.

\item{\texttt{out.type}}: A character vector specifying what kind of
information the simulator should remember from each period.  There are
five basic types, any combination of which can be specified:

\begin{itemize}

\item{\bf{basic}}: Saves information on the equity and size of the
portfolio, the turnover in both the portfolio and the universe, and
summary information on the portfolio's performance for each period.

\item{\bf{detail}}: Saves performance details for all stocks in the
portfolio at each period.

\item{\bf{exposures}}: Saves exposures for each period.

\item{\bf{contributions}}: Saves contributions for each period.

\item{\bf{trades}}: Saves the list of trades performed for each
period.

\end{itemize}

In addition, there are three preset configurations of these types that
can be specified in the \texttt{out.type} slot:

\begin{itemize}

\item{\bf{all}}: Saves all five types of data.

\item{\bf{default}}: Saves the basic result information and the trades
for each period.  This mode also saves exposures if an
\texttt{exp.var} has been specified and contributions if a
\texttt{contrib.var} has been specified.  This is the default behavior
for the simulator.

\item{\bf{lean}}: Same as the ``default'' type above, except trades
are not saved.

\end{itemize}

\end{itemize}


\subsection{Interfaces}

The flexibility of the \texttt{portfolioSim} package stems from three
virtual classes which serve as interfaces that the user can implement
to create a customized simulation.  Each of these interfaces deals
with a different part of the simulator, but multiple interfaces can be
designed to work together.  The first two interfaces deal with the two
primary types of input required by the simulator.

\begin{itemize}

\item{\bf{simDataInterface}}: All the raw data used in the simulation
is accessed through the data interface.  This is interface is supplied
to the simulator by the user, and must be an object containing the
virtual class \texttt{simDataInterface}.  The actual data can be
stored in as an \texttt{R} object in the interface itself, or the
interface can query some outside database to obtain information on a
given period.  An implementation of the \texttt{simDataInterface}
class must contain the method \texttt{getSimData}, whose purpose it is
to retrieve the raw data and transform it into a \texttt{simData}
object.

A \texttt{simData} object contains all the information the simulator
will need during a single period.  This informationo is stored as a
data frame, with each row representing a unique security.  The simulator
itself requires every \texttt{simData} object to include the following
columns:

\begin{itemize}

\item{\texttt{period}}: The period to which the data in this
\texttt{simData} object corresponds.  Since a \texttt{simData} object
contains data from only one period, all the values for period should
be identical.  This value should also match one of the periods in the
\texttt{periods} slot of the \texttt{portfolioSim} object.

\item{\texttt{id}}: A unique identifier for each security.

\item{\texttt{start.price}}: The price of this security at the
beginning of the period.

\item{\texttt{end.price}}: The price of this security at the end of
the period.

\item{\texttt{ret}}: Total return for this period.

\item{\texttt{universe}}: A logical vector indicating whether
securities are in the universe of investable stocks for this period.

\item{\texttt{volume}}: The daily trading volume of this security.

\end{itemize}

Since the simulator requires all this data, it must either be included
in the raw data retrieved by the \texttt{simDataInterface}, or the
interface must have some means of calculating it.  The other interfaces
used in a simulation might require additional columns to appear in the
\texttt{simData} object.  For example, the trades interface included
with the \texttt{portfolioSim} package, \texttt{stiFromSignal},
requires some ``signal'' column in the data (see section 4).  When
designing multiple interfaces together, it can be useful to require
additional columns in \texttt{simData} objects.

A very simple data interface, the \texttt{sdiDf} class, is included
with the \texttt{portfolioSim} package.  This interface stores all the
raw data in a data frame, and the \texttt{getSimData} method simply
subsets the data frame by period.  Because the interface performs no
additional work to format the data, all the columns required
throughout the simulation must be included in the raw data when using
the \texttt{sdiDf} interface.

\item{\bf{simTradesInterface}}: The trades interface respresents the
investment strategy the simulator uses to select trades.  This
interface, stored in the \texttt{trades.interface} slot of the
simulator, determines the trades to perform at each period and returns
them as an object of class \texttt{simTrades}.  The process by which
these trades are determined takes place entirely within the
\texttt{getSimTrades} method of the interface, so the possible trading
strategies are limitless.  The only requirement is that the method
return a \texttt{simTrades} object.  This class is basically a wrapper
for an object of the \texttt{trades} class contained in the
\texttt{portfolio} project.  A \texttt{trades} object is simply a data
frame with columns \texttt{id}, \texttt{side}, and \texttt{shares}
indicating which stock to trade, how to trade it (buy, sell, short, or
cover), and how much of it to trade.

The trades interface \texttt{stiFromSignal} is included in the
\texttt{portfolioSim} package.  This class is designed to take full
advantage of the \texttt{tradelist} features found in the
\texttt{portfolio} package.  It can be used to test the effectiveness
of any kind of signal one might use to make investments, anything from
professional stock analyses to one-day returns.  See section 4 for a
detailed introduction to using the \texttt{stiFromSignal} interface.

\end{itemize}

The third interface allows the user to customize the output of the
simulator.  The \texttt{out.type} slot of the \texttt{portfolioSim}
class allows for a considerable amount of flexibility in the type of
data the simulator can return.  However, in some cases the user might
want to save out some information beyond what the five basic result
types allow.  The summary interface gives the user this ability.
Because the output types built into the result classes are sufficient
for the most common types of simulations, this interface is optional.

\begin{itemize}

\item{\bf{simSummaryInterface}}: The summary interface is the most
open-ended of the interfaces.  At each period, the simulator passes to
the summary interface a snapshot of the current portfolio, along with
the \texttt{simData} and \texttt{simTrades} objects for the period.
What the summary interface does with this data is entirely up to the
user.  Generally, the interface will save some sort of statistical
data on the portfolio at the current period.  This data can either be
stored in the interface itself, or it can be saved out like the other
simulation results.  The \texttt{updateSummary} method is responsible
for computing whatever summary data the user wants to keep and storing
it to be accessed later.

The summary interface requires a second method, called the
\texttt{summary} method.  This method is called from within the
\texttt{summary} method of the \texttt{simResult} class, and is
responsible for accessing and displaying whatever data the interface
is storing.  Other accessor or helper methods for the interface can be
provided at the user's discretion.

\end{itemize}


\subsection{Classes for storing simulator results}

The \texttt{runSim} method of class \texttt{portfolioSim} stores the
results of the simulation in a hierarchy of results classes, outlined
below.  The user most often interacts with objects of the
\texttt{simResult} class, which provides methods for displaying
statistical and summary information on the simulation.

\begin{itemize}

\item{\texttt{instantData}}: Stores information about the portfolio at
a single point in time.  Specifically, this class stores information
on the current holdings in the portfolio, the size and equity of the
long and short sides of the portfolio, and any exposures if the
simulator has some \texttt{exp.var} specified.

\item{\texttt{periodData}}: Stores information on a single period in
the simultion.  This includes turnover in the portfolio and in the
universe, all the trades done in a given period, the performance of
the portfolio during that period, and contributions to that
performance if a \texttt{contrib.var} is specified.

\item{\texttt{simResultSinglePeriod}}: For each period in the
\texttt{periods} slot of the simulator, one object of class
\texttt{simResultSinglePeriod} is created.  This class contains one
slot for an object of class \texttt{periodData}, and two slots for
objects of class \texttt{instantData}, one representing the portfolio
at the start of the period and one at the end.

\item{\texttt{simResult}}: Stores information on the simulation as a
whole, including annual frequency of the periods, the type of results
stored, and any error messages that occured during the simulation.
Most importantly, \texttt{simResult} objects contain a list of the
\texttt{simResultSinglePeriod} objects for each period in the
simulation.  In addition, it stores the final version of the summary
interface, if one is specified.

\end{itemize}

For more information on accessing the results of a simulation, see
section 6.


\section{Using the \texttt{stiFromSignal} interface}

The \texttt{stiFromSignal} trades interface included in the
\texttt{portfolioSim} package requires some numeric ranking of stocks.
It then generates trades intended to maintain a portfolio with the
highest ranked stocks on the long side and the lowest ranked stocks on
the short side.  This is only one trading strategy, but it is very
useful for testing the accuracy of any model that makes quantifiable
predictions of a stock's future performance.  Testing such models is
one of the most common reasons for conducting a backtest.

The core of the \texttt{stiFromSignal} interface is in the generation
of a \texttt{tradelist} object, part of the \texttt{portfolio}
package.  The mechanism by which \texttt{tradelist} calculates trades
lies beyond the scope of this article.\footnote{For a detailed
introduction to \texttt{tradelist}, see Enos, Gerlanc, and Kane,
Trading with the portfolio package.}  This section is intended to
allow a user unfamiliar with \texttt{tradelist}s to use the
\texttt{stiFromSignal} interface to conduct simulations.

An object of class \texttt{stiFromSignal} has the following slots:

\begin{itemize}

\item{\texttt{in.var}}: A variable to be used as the ``signal'' by
which stocks are ranked.  The \texttt{in.var} must be a column of the
data frame stored in the \texttt{simData} objects that the data
interface returns.

\item{\texttt{type}}: The type of weight formation to be used when
forming the portfolio (see the documentation for the
\texttt{portfolio} package for details).  The default is ``equal'',
which results in an equal-weighted portfolio.

\item{\texttt{size}}: The size of the portfolio to be formed; can
either specify the number of securities per side, or can be relative
to the total number of securities.  The default is ``quintile'',
meaning each side of the portfolio will consist of one fifth of the
stocks for which the \texttt{in.var} provides rankings.

\item{\texttt{sides}}: The sides to be contained in the portfolio.
Can be ``long'', ``short'', or c(``long'', ``short'').

\item{\texttt{equity}}: The total equity of the portfolio.

\item{\texttt{target}}: An environment in which the interface stores
its target portfolio between periods.  Because of the restrictions
placed on trades by \texttt{tradelist}, very often it is not possible
to attain the portfolio specified by the signal during a single
trading period.  Therefore, the target portfolio is saved so that the
interface will continue trading towards that target in subsequent
periods.

\item{\texttt{rebal.on}}: A vector of periods during which the target
portfolio is to be rebalanced.  These periods must correspond to the
periods in the \texttt{periods} slot of the \texttt{portfolioSim}
object.  Rebalancing is simply the process of forming a target
portfolio from the \texttt{in.var}.  The interface rebalances
automatically if there is no saved target.  Otherwise, it will
continue trading to the target portfolio until it reaches a
\texttt{rebal.on} period.

\item{\texttt{trading.style}}: The style of trading to use. Possible
styles are:

\begin{itemize}

\item{\bf{immediate}}: The default trading style, this style returns
trades that immediately transform the current portfolio into the
target portfolio.  This style is the simplest, but also the most
unrealistic since it overrides many of the \texttt{tradelist} features
intended to simulate actual trading.

\item{\bf{percent.volume}}: Returns trades for a maximum of 15 percent
of a stock's daily trading volume; otherwise the same as ``immediate''.

\item{\bf{15.pct.vol.to.equity}}: The most realistic trading style,
making use of all the features in \texttt{tradelist}, such as
trade-cost adjustment.  This style requires the column
\texttt{md.volume.120.d} in the \texttt{simData} object, and allows
for sorts if the column \texttt{alpha.6} appears.

\end{itemize}

\item{\texttt{chunk.usd}}: The maximum chunk size, in U.S. dollars,
into which \texttt{tradelist} breaks up possible trades.

\item{\texttt{turnover}}: The maximum turnover allowed in the
portfolio per period.  Defaults to infinity, placing no restriction on
turnover.

\end{itemize}

The \texttt{getSimTrades} method of \texttt{stiFromSignal} consists of
two basic steps.  First, it generates a target portfolio.  If the
portfolio is not being rebalanced, this means simply retrieving the
saved target.  If the portfolio needs to be rebalanced, the
\texttt{portfolio} package is used to generate a new portfolio from
the \texttt{in.var}.  The kind of portfolio created depends on the
\texttt{type}, \texttt{size}, \texttt{sides}, and \texttt{equity}
slots.  Second, the current portfolio and the target portfolio are
used to create a new \texttt{tradelist} object.  What kinds of trades
are generated depends on the \texttt{trading.style} specified, along
with the \texttt{chunk.usd} and \texttt{turnover} slots.  The
interface then returns the \texttt{trades} object contained in the
\texttt{tradelist}, and saves the target portfolio to the
\texttt{target} environment to be used in the next period.


\section{A multi-period example}

Starmine is a San Fransisco based research company that creates
rankings of stocks based on predicted future earnings.  One such
ranking is the StarMine Indicator, which ranks stocks on a scale of 1
to 100, with 100 being the highest.\footnote{See www.starmine.com} We
can use the \texttt{stiFromSignal} interface to test the accuracy of
the StarMine Indicator.  If there is indeed a positive correlation
between StarMine rankings and returns, we should expect a long-short
portfolio formed by buying the highest ranked stocks and shorting the
lowest ranked stocks to yield high returns.  How well our portfolio
performs should give us some idea of whether we wish to use the
StarMine Indicator to make investment decisions.

The data set \texttt{starmine.sim} included in the
\texttt{portfolioSim} package contains StarMine Indicator rankings for
stocks from January 31, 1995 to November 30, 1995.  In this data set,
the rankings are updated monthly.  All the other data we need to run a
simulation are also included in the data set.

\begin{Schunk}
\begin{Sinput}
> data(starmine.sim)
> names(starmine.sim)
\end{Sinput}
\begin{Soutput}
 [1] "id"              "date"            "name"           
 [4] "country"         "sector"          "cap.usd"        
 [7] "size"            "smi"             "fwd.ret.1m"     
[10] "fwd.ret.6m"      "price.usd"       "prior.close.usd"
[13] "volume"          "ret.1m"         
\end{Soutput}
\end{Schunk}

The first step in constructing our \texttt{portfolioSim} object is to
create the periods data frame.  Looking at the \texttt{date} column,
we can easily construct a data frame that looks like this:

\begin{Schunk}
\begin{Soutput}
       period      start        end
1  1995-03-31 1995-03-31 1995-06-30
2  1995-06-30 1995-06-30 1995-07-31
3  1995-07-31 1995-07-31 1995-08-31
4  1995-08-31 1995-08-31 1995-09-30
5  1995-09-30 1995-09-30 1995-10-31
6  1995-10-31 1995-10-31 1995-11-30
7  1995-11-30 1995-11-30 1995-01-31
8  1995-01-31 1995-01-31 1995-02-28
9  1995-02-28 1995-02-28 1995-04-30
10 1995-04-30 1995-04-30 1995-05-31
11 1995-05-31 1995-05-31 1995-12-31
\end{Soutput}
\end{Schunk}

Special attention must be paid to the \texttt{period} column.  These
are the labels used at every level of the simulator to identify the
current period.  Any time the data or trades interfaces need to refer
to a period, it must be identical to one of the periods in this
column.  On one other hand, these labels can also be more abstract
than the \texttt{start} and \texttt{end} columns; it is perfectly
acceptable to simply number the periods, so long as the numbering
system is consistent throughout the simulator.

The next step is to set up the data interface.  Note that of the
information we need for our \texttt{simData} objects (id, period,
start price, end price, volume, return, and universe), all except for
``universe'' are contained in the data set.  However, the column names
do not match those required by the simulator.  If we expected to run
many simulations with data formatted in this same way, it would
probably be worth our time to write a new \texttt{simDataInterface}
which, as part of its \texttt{getSimData} method, would change all the
names for us and add a \texttt{universe} column.  However, for the
purposes of this example, it is easier to simply make these changes
manually and use the \texttt{sdiDf} interface included with the
package.

We already have \texttt{id} and \texttt{volume} columns in the data
set.  Our \texttt{period} column is clearly \texttt{date}, so we can
rename this first.

\begin{Schunk}
\begin{Sinput}
> starmine.sim$period <- starmine.sim$date
\end{Sinput}
\end{Schunk}

It it important to note that while the periods are spaced at monthly
intervals, most of the columns in the data set refer to single days,
not months.  We must therefore consider carefully how we would go
about trading in the real world, if we had only the information
contained in this data set.  Essentially, we are only allowed to trade
on one day of every period, specifically the last trading day of the
month.  We therefore take the closing price of the day before the last
trading day of the month as the \texttt{start.price}, and the closing
price for the last trading day of the month as the \texttt{end.price}.

\begin{Schunk}
\begin{Sinput}
> starmine.sim$start.price <- starmine.sim$prior.close.usd
> starmine.sim$end.price <- starmine.sim$price.usd
\end{Sinput}
\end{Schunk}

We now need to select a column for the returns the simulator will use
to calculate the portfolo's performance.  The simulator uses the
portfolio at the start of each period, before any trades are made, to
calculate the performance for that period.  In other words, we assume
that it takes the entire span of a period for us to trade to a new
portfolio.  So even though we are trading on January 31, 1995, the
simulator assumes that we do not get the portfolio resulting from those
trades until the end of the period, on February 28, 1995.  This is
somewhat counter-intuitive, because we want to use returns from the
month of February to calculate the performance of the portfolio that
we traded to at the end of January (remember that we actually do all
our trading on January 31, and then hold that portfolio throughout the
rest of the period).  To correct this, we use one month backward
returns so that the performance for the period beginning on February
28 is calculated using the portfolio we formed on January 31 and the
returns from the month of February.

\begin{Schunk}
\begin{Sinput}
> starmine.sim$ret <- starmine.sim$ret.1m
\end{Sinput}
\end{Schunk}

Finally, we need to add a \texttt{universe} column to the data set.
The \texttt{universe} column is a logical indicating which of the
stocks in the \texttt{simData} object the simulator will be allowed to
trade.  The data interface can define this flag in any number of ways.
For the purposes of this example, we will assume that all the stocks
for which we have data exist in the investable universe for that
period.

\begin{Schunk}
\begin{Sinput}
> starmine.sim$universe <- TRUE
\end{Sinput}
\end{Schunk}

Now that we have our data correctly formatted, we can create a new
object of class \texttt{sdiDf} to be used as the data interface in our
simulation.

\begin{Schunk}
\begin{Sinput}
> data.interface <- new("sdiDf", data = starmine.sim)
\end{Sinput}
\end{Schunk}

Again, the process of reformatting our input data would quickly become
tedious if we were running multiple simulations like this.  It would
be quite easy to create a modified version of \texttt{sdiDf} to do all
this work for us.

After the data interface has been created, we move on to the trades
interface.  In this example we will use the \texttt{stiFromSignal}
interface, which is specifically designed for testing the accuracy of
a numeric ranking such as the StarMine Indicator.

We set the \texttt{in.var} to be \texttt{smi}, the ranking that the
simulator will use to form the target portfolio.  For the sake of
simplicity, and because we have a relatively small number of periods,
we want the interface to rebalance the target portfolio at every
period, so we simply pass the \texttt{period} column from the
\texttt{periods} data frame to \texttt{rebal.on} slot.  Because of the
large number of stocks in our data set, we set the size of the
portfolio to ``decile'' and the total equity to one million
U.S. dollars.  All the other slots keep their default settings.  We do
not worry about setting a target environment; the initialize method of
\texttt{stiFromSignal} will do this for us.

\begin{Schunk}
\begin{Sinput}
> trades.interface <- new("stiFromSignal", in.var = "smi", 
+     size = "decile", equity = 1000000, rebal.on = periods$period)
\end{Sinput}
\end{Schunk}

This interface will return trades for creating and maintaining a
long-short, equal-weighted portfolio, with the long side formed from
the ten percent of stocks receiving the highest StarMine Indicator
rankings, and the short side formed from the ten percent receiving the
lowest rankings.  The total equity of the portfolio will be kept at
\$1,000,000.  Trades will be selected based on the ``immediate''
trading style, meaning that at each period, the list of trades
returned will shift all of our holdings to match the target portfolio.
This target will be rebalanced at each period to match the StarMine
Indicator rankings for that month.

Finally, we are ready to construct a \texttt{portfolioSim} object to
contain all these elements of the simulator:

\begin{Schunk}
\begin{Sinput}
> ps <- new("portfolioSim", periods = periods, freq = 12, data.interface = data.interface, 
+     trades.interface = trades.interface, fill.volume.pct = Inf, 
+     out.loc = "out_dir_2", out.type = "lean")
\end{Sinput}
\end{Schunk}

Because we are using monthly periods, we set \texttt{freq} to be 12.
Because we do not specify any \texttt{start.holdings}, the simulator
will begin with an empty portfolio.  The \texttt{fill.volume.pct} is
set to infinity so that all the trades returned from the trades
interface will be filled.  The \texttt{out.loc} can be an existing
directory, or a non-existent directory in which case the simulator
will create it.  Our \texttt{out.type} is ``lean'', so only the
``basic'' type of results will be saved out, since we have no
\texttt{exp.var} or \texttt{contrib.var}.  We also have no summary
interface to specify, so the simulator will save no additional
information.

Once our \texttt{portfolioSim} object is created, we simply call the
\texttt{runSim} method to conduct the simulation.

\begin{Schunk}
\begin{Sinput}
> result <- runSim(ps, verbose = FALSE)
\end{Sinput}
\end{Schunk}

The \texttt{runSim} method steps through each period of simulation.
If first uses the data interface to retrieve a \texttt{simData} object
containing the data for that period.  Second, the trades interface is
called to generate the trades to be conducted for that period.
Finally, the \texttt{performance} and \texttt{expose} methods of the
\texttt{portfolio} class are used to update the holdings based on the
trades returned from the trades interface.  A
\texttt{simResultSinglePeriod} object is created for each period, and
stored in a master \texttt{simResult} object.  The results are also
saved out after each period to the \texttt{out.loc} directory
specified in the \texttt{portfolioSim} object.

\section{Analyzing the simulation results}

The \texttt{summary} method of a \texttt{simResult} object can be used
to obtain a quick summary of the simulation results.

\begin{Schunk}
\begin{Sinput}
> summary(result)
\end{Sinput}
\begin{Soutput}
Simulation summary:


Start:   1995-01-31
End:     1995-11-30

                                    profit   return
---------------------------------------------------
Total:                             218,063     23.2 %  
Sharpe:                                         3.1
---------------------------------------------------
Mean return:                                    1.8 %  
Mean return (ann):                             21.3 %  
Volatility:                                     2.0 %  
Volatility (ann):                               6.8 %  
---------------------------------------------------
Best period:                        55,760      6.3 %   (1995-10-31)
Worst period:                       -4,234    -43.8 bps (1995-05-31)
Worst drawdown:                      9,432     78.7 bps (1995-04-30 to 1995-05-31)
---------------------------------------------------
Mean size:                             522
Mean equity:                       838,058
Mean gross equity:               1,676,116
Universe turnover:                       0
Turnover:                       26,352,842
Mean turnover:                   2,196,070
Mean turnover (ann):            26,352,842
Holding period (mth):                  1.5
---------------------------------------------------
Mean NA weights:                        35
Mean NA returns:                        32
\end{Soutput}
\end{Schunk}

This display summarizes the data saved by the simulator when using the
``basic'' \texttt{out.type}.  From this summary, we can quickly
observe that our investment strategy based on the StarMine Indicator
yielded a profit of \$218,063, or 23.2\% of our original \$1,000,000
investment.  Across our monthly periods, the mean return was 1.8\%.
The best month for our portfolio was October, while our worst returns
came in May.  Overall, it seems like our strategy would have worked
very well during 1995.  While such strong returns are encouraging, we
would have to compare these figures to the behavior of the market as a
whole during that year, in order to gauge how well the model performed
relative to the rest of the universe.

If the simulation is conducted with the ``detail'' \texttt{out.type},
the summary method prints two lists of the best and worst performing
stocks over the entire simulation.  The first list is based on profit,
while the second list is based on contribution.


\begin{Schunk}
\begin{Soutput}
Simulation summary:

Top/bottom performers by profit:

           id profit
799  45031210  -5973
357  16961710  -5127
284  11157210  -3930
1001 52465R10  -3292
1338 64108110  -2618
2414 G2847110  -2557
317  12738710   4032
372  17275510   4227
68   01877H10   4393
1742   680492   4574
2417 N5424G10   4720
2000 72003530   5322

Top/bottom performers by contribution (%):

           id contrib
799  45031210   -0.60
357  16961710   -0.49
284  11157210   -0.40
1001 52465R10   -0.33
2056 75886F10   -0.25
1338 64108110   -0.25
317  12738710    0.39
372  17275510    0.40
68   01877H10    0.43
1742   680492    0.45
2417 N5424G10    0.46
2000 72003530    0.53
\end{Soutput}
\end{Schunk}

The saved results from a previous simulation can be loaded into a new
\texttt{R} session by using the \texttt{loadIn} method of class
\texttt{simResult}:

\begin{Schunk}
\begin{Sinput}
> result <- loadIn(new("simResult"), in.loc = "out_dir_2")
\end{Sinput}
\end{Schunk}

All the analysis methods previously discussed can also be called on a
\texttt{simResult} object loaded from previously saved results.


\section{Running \texttt{portfolioSim} with contributions and exposures}

Note: this section of the document is unfinished.

The \texttt{portfolioSim} object has two slots, \texttt{contrib.var}
and \texttt{exp.var}, for specifying contribution and exposure
variables.  These are variables found in the data slot of
\texttt{simData} objects that the interface returns.  The simulator
can thus analyze the exposures and contributions to the portfolio
across other variables.

For example, suppose we want to look at the exposures to our portfolio
from price and country.  We add these two variables, both included in
our data set, to the \texttt{exp.var} slot of the simulator, and
change the \texttt{out.type} to ``exposures''.  (The ``default'' or
``lean'' types will automatically save exposures data if an
\texttt{exp.var} is specified, but in this case we are only interested
in looking at exposures.)

%% <<echo=TRUE>>=
%% ps <- new("portfolioSim", periods = periods, freq = 12,
%% data.interface = data.interface, trades.interface = trades.interface,
%% exp.var = c("start.price", "country"), fill.volume.pct = Inf,
%% out.loc = "output_directory", out.type = "exposures")
%% @ 

%% <<echo=TRUE, results=hide>>=
%% result <- runSim(ps, verbose = FALSE)
%% @ 

%% <<echo=TRUE>>=
%% summary(result)
%% @ 

Likewise, to see the contributions to our portfolio from different
sectors, we simply specify ``sector'' as a \texttt{contrib.var} and
set \texttt{out.type} to ``contributions''.

%% <<echo=TRUE, results=hide>>=
%% ps <- new("portfolioSim", periods = periods, freq = 12,
%% data.interface = data.interface, trades.interface = trades.interface,
%% contrib.var = "sector", fill.volume.pct = Inf,
%% out.loc = "output_directory", out.type = "contributions")
%% @ 

%% <<echo=TRUE>>=
%% summary(result)
%% @ 

For an introduction to contributions and exposures, see the
documentation in the \texttt{portfolio} package.

\section{Conclusion}

Simulating investments over some period in the past is often a time
consuming and data-intensive process.  The \texttt{portfolioSim}
package automates the process of running such a simulation, by
allowing the user to specify in advance all the details of the
portfolio to be maintained, the investment strategy by which to
maintain it, and the results from the simulation to be saved.  The
flexibility of the \texttt{portfolioSim} package stems from its
interfaces, which can be customized to work with virtually any type of
input data and any method of investment.  Together with the
\texttt{portfolio} package, \texttt{portfolioSim} provides investors
with a powerful and adaptable set of tools for managing their
investments.


\end{document}
